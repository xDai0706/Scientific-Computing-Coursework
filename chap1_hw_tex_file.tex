\documentclass[a4paper,12pt,answers,addpoints]{exam}

% Packages:
% ============================================================
\usepackage[utf8]{inputenc}
\usepackage{amsmath,amsfonts,amssymb,bm,mathtools}
%\usepackage{physymb}
\usepackage{braket}
\usepackage{graphicx}
\usepackage{epsfig}
\usepackage{afterpage}
\usepackage{caption}
\usepackage[makeroom]{cancel}
\usepackage[hang,nooneline]{subfigure}
%\usepackage{fancyhdr}

%\pagestyle{fancy}
%\fancyhf{}

\rfoot{Page \thepage \hspace{1pt} of \numpages}
% New commands:
% ============================================================
\newlength\dlf
\newcommand\alignedbox[2]{
	% #1 = before alignment
	% #2 = after alignment
	&
	\begingroup
	\settowidth\dlf{$\displaystyle #1$}
	\addtolength\dlf{\fboxsep+\fboxrule}
	\hspace{-\dlf}
	\boxed{#1 #2}
	\endgroup
}
\newcommand*{\QEDA}{\hfill\ensuremath{\blacksquare}}%
\newcounter{fig}
\newcommand{\lbfig}[1]{\refstepcounter{fig}           .
	\label{#1} }

%opening
\title{Introduction to Scientific Computing  \textbf{Fall 2022}}
\author{}

% Nice commands:
%\pagestyle{headandfoot}
\runningheadrule
\firstpageheader{\textbf{Math 551 Section 02}}{\textbf{\hspace{1.3cm} }}{\textbf{Fall 2022}}
\runningheader{\textbf{Math 551}}{\textbf{Chapter 1 Homework Set}}
{Fall 2022}


%\thispagestyle{empty}

\begin{document}
	%\begin{center}
	%\textsc{\large university of massachusetts amherst\\
	%department of mathematics and statistics\\
	%\textbf{Introduction to Scientific Computing}
	%\\\textbf{Math 551 Section 3}, SPRING 2020
	%}
	%\end{center}
	\begin{center}
		\textbf{Chapter 1 Homework due Thursday, February 23rd, 2023 at 11:59 PM}\\
		%\textbf{Solutions}
	\end{center}
	%\begin{center}
	%\includegraphics[height=2.00cm,width=2.00cm]{umass.eps}
	%\end{center}
	
	%\maketitle
	
	\begin{center}
		\fbox{\fbox{\parbox{6in}{
					\textbf{Notes}: \textit{Try to answer all the questions by demonstrating all the steps of your calculations. %Please put your long answers to each question 1(a), 1(b), 1(c), 2, 3(a), 3(b) and 4 on a different page.
						Please submit your homework on Gradescope. This homework assignment covers Section 1.1, 1.2 $\&$ 1.3. Answer all the questions by demonstrating all the steps of your calculations.
						%, and return your answers (including this homework sheet) back to the instructor as a whole package \textbf{stapled}. This homework assignment covers Section 1.1, 1.2 $\&$ 1.3. %Make sure that you copy   your solutions and keep them in a binder so that you may easily reference that when you are studying for an exam.
		}}}}
	
\vspace{0.25in}
\makebox[\textwidth]{ \textbf{Name (Last, First)}:\enspace\hrulefill\textbf{Student ID \#:}\rule{0.8in}{0.25pt}}
\end{center}
\vspace{0.08in}
\begin{questions}


% ---------------------------------------------


% Question: 1
% -------------------------------------------------------------------
\question Recall Taylor's theorem from Calculus:
``Assume a function $f(x)$ that has $k+1$ derivatives in an interval 
$\left[a,b\right]$, or simply, $f\in C^{k+1}\left[a,b\right]$ and 
$x_{0}\in \left[a,b\right]$. Then, for every $x\in \left[a,b\right]$,
$\exists\,\xi$ between $x_{0}$ and $x$ such that 
%
\begin{equation} 
f(x) = \underbrace{\sum_{n=0}^{k}\frac{f^{(n)}(x_{0})}{n!}\left(x-x_{0}\right)^{n}}_{P_{k}(x)}%
+ \underbrace{\frac{f^{(k+1)}(\xi)}{(k+1)!}\left(x-x_{0}\right)^{k+1}}_{R_{k}(x)},
\label{taylor_thm}
\end{equation}
%
where $P_{k}(x)$ is called the $k$th Taylor polynomial for $f$ around $x_{0}$ 
and $R_{k}(x)$ is called the remainder, or truncation error. Note that 
$$\lim_{k\to \infty}P_{k}(x)$$ gives the Taylor series for the 
same function $f$ about $x=x_{0}$ and also a function $f$ is analytic in 
$\left(a,b\right)$ if the Taylor series equals $f$ for all $x\in\left(a,b\right)$.
Finally, the Taylor series around $x=x_{0}\equiv 0$ is called MacLaurin series.
\begin{parts}
\part (8 points) Find $P_{1}(x)$, $P_{2}(x)$ and $P_{3}(x)$ around $x_{0}=0$ if $f(x)=x^2-4x+3$.
How $P_{3}(x)$ is related to $f(x)$?
\part (7 points) Same as part (a) but consider $x_{0}=1$.
\part (5 points) Given a polynomial $f(x)$ with degree $m$, what can 
you say about $f(x)-P_{k}(x)$ for $k\ge m$?
\end{parts}



% Question: 2
% -------------------------------------------------------------------
\question[15] Given the function $f(x)=\cos{x}$, find both $P_{2}(x)$
and $P_{3}(x)$ about $x_{0}=0$, and use them to approximate
$\cos{(0.1)}$. Show that in each case the remainder term provides
an upper bound for the true (absolute) error.


% Question: 3
% -------------------------------------------------------------------
\question Consider the function $f(x)=e^{x}$.
%
\begin{parts}
\part (10 points) Find the MacLaurin series of the function $f(x)=e^{x}$, 
i.e., the Taylor series about $x_{0}=0$ (write separately $P_{k}(x)$
and $R_{k}(x)$),
\part (10 points) Find a minimum value of $k$ necessary for $P_{k}(x)$ to
approximate $f(x)$ to within $10^{-6}$ on the interval $[0,0.5]$
(here, you must use the remainder term).
\end{parts}


% Question: 4
% -------------------------------------------------------------------
\question (5 points) Given that
\[
R(x)=\dfrac{|x|^6}{6!}e^{\xi},
\]
for $x\in\left[-\frac{1}{2}, \frac{1}{2}\right]$, where $\xi$ is between $x$ and $0$, find an upper bound for $|R|$, valid for all $x\in \left[-\frac{1}{2}, \frac{1}{2}\right]$, that is independent of $x$ and $\xi$.

% Question: 5
% -------------------------------------------------------------------

\question (10 points) Use Taylor’s Theorem to show that
\[
(1+x)^{-1}=1-x+x^2+\mathcal{O}(h^3)
\]
for some $h>0$ sufficiently small.


% Question: 6
% -------------------------------------------------------------------

\question (10 points) Use Taylor expansions for $f(x\pm h)$ to derive an $\mathcal{O}(h^2)$ accurate approximation to the second derivative $f''(x)$ using $f(x)$ and $f(x\pm h)$. Provide all the details of the error estimate.
\newline
\textit{Hint}: Go out as far as the fourth derivative term, and then add the two expansions.


	
% Question: 7
% -------------------------------------------------------------------
\question MATLAB

\begin{parts}
\part (10 points) Using the $m$-file \textit{taylor\_prob3.m}, plot the function $f(x)=e^x$ and its MacLaurin approximations $p_2$, $p_4$ and $p_6$. Run the script and show the graphs.
\part (10 points) Modify the $m$-file \textit{abs\_errors.m} with initial step-size $h=0.15$ and $N=10$, evaluate and printout the absolute errors.

\end{parts}




\end{questions}


\end{document}
